
\documentclass[11pt]{article}
\usepackage{times}
\usepackage{url}
\usepackage{latexsym}
\usepackage{natbib}
\usepackage{geometry}
\usepackage{indentfirst}
\usepackage{setspace}
\usepackage{lineno}
\usepackage{natbib}

\usepackage[colorlinks,
linkcolor=black,
anchorcolor=blue,
citecolor=blue]{hyperref}
\setlength{\parindent}{1em}
\setlength{\parskip}{0.5em}
\setlength\linenumbersep{1.6cm}
\setlength\columnsep{0.6cm}
\newlength\titlebox 
\setlength\titlebox{5cm}
\geometry{a4paper,left=2.5cm,right=2.5cm,top=2.5cm,bottom=2.5cm}

\def\href#1#2{{#2}}
\renewcommand\cite{\citep}  % to get "(Author Year)" with natbib    
\newcommand\shortcite{\citeyearpar}% to get "(Year)" with natbib    
\newcommand\newcite{\citet} % to get "Author (Year)" with natbib



\begin{document}
\title{Research Proposal for PhD Joint Training Program}
\author{Huchong Ling \\
  Northeastern University \\
  {\tt xxxxxx@xxx.com} \\}
\date{2022-04-20}
\maketitle
\pagenumbering{roman}
\tableofcontents
\newpage
\pagenumbering{arabic}
\section{Introduction}
The written text that we interact with on an everyday basis—news articles, emails, social media, books is the product of a profoundly social phenomenon with people at its core. With few exceptions, all of the text we see is written by people, and others constitute its audience. A vast amount of the content itself is centered on people: news (including classic NLP corpora such as the Wall Street Journal and the New York Times) details the roles of actors in current events, social media (including Twitter and Facebook) documents
the actions and attitudes of friends, and books chronicle the stories of fictional characters and real people alike~\cite{serban2016building}.

\section{Persona Characterization and Variation1}
\subsection{Background and Motivation}
The written text that we interact with on an everyday basis—news articles, emails, social media, books is the product of a profoundly social phenomenon with people at its core. With few exceptions, all of the text we see is written by people, and others constitute its audience. A vast amount of the content itself is centered on people: news (including classic NLP corpora such as the Wall Street Journal and the New York Times) details the roles of actors in current events, social media (including Twitter and Facebook) documents
the actions and attitudes of friends, and books chronicle the stories of fictional characters and real people alike~\cite{serban2016building}.
\subsection{Research Purpose}
The written text that we interact with on an everyday basis—news articles, emails, social media, books is the product of a profoundly social phenomenon with people at its core. With few exceptions, all of the text we see is written by people, and others constitute its audience. A vast amount of the content itself is centered on people: news (including classic NLP corpora such as the Wall Street Journal and the New York Times) details the roles of actors in current events, social media (including Twitter and Facebook) documents
the actions and attitudes of friends, and books chronicle the stories of fictional characters and real people alike~\cite{serban2016building}.
\subsection{Methodology}
The written text that we interact with on an everyday basis—news articles, emails, social media, books is the product of a profoundly social phenomenon with people at its core. With few exceptions, all of the text we see is written by people, and others constitute its audience. A vast amount of the content itself is centered on people: news (including classic NLP corpora such as the Wall Street Journal and the New York Times) details the roles of actors in current events, social media (including Twitter and Facebook) documents
the actions and attitudes of friends, and books chronicle the stories of fictional characters and real people alike~\cite{serban2016building}.
\section{Persona Characterization and Variation2}
\subsection{Background and Motivation}
The written text that we interact with on an everyday basis—news articles, emails, social media, books is the product of a profoundly social phenomenon with people at its core. With few exceptions, all of the text we see is written by people, and others constitute its audience. A vast amount of the content itself is centered on people: news (including classic NLP corpora such as the Wall Street Journal and the New York Times) details the roles of actors in current events, social media (including Twitter and Facebook) documents
the actions and attitudes of friends, and books chronicle the stories of fictional characters and real people alike~\cite{serban2016building}.
\subsection{Research Purpose}
The written text that we interact with on an everyday basis—news articles, emails, social media, books is the product of a profoundly social phenomenon with people at its core. With few exceptions, all of the text we see is written by people, and others constitute its audience. A vast amount of the content itself is centered on people: news (including classic NLP corpora such as the Wall Street Journal and the New York Times) details the roles of actors in current events, social media (including Twitter and Facebook) documents
the actions and attitudes of friends, and books chronicle the stories of fictional characters and real people alike~\cite{serban2016building}.
\subsection{Methodology}
The written text that we interact with on an everyday basis—news articles, emails, social media, books is the product of a profoundly social phenomenon with people at its core. With few exceptions, all of the text we see is written by people, and others constitute its audience. A vast amount of the content itself is centered on people: news (including classic NLP corpora such as the Wall Street Journal and the New York Times) details the roles of actors in current events, social media (including Twitter and Facebook) documents
the actions and attitudes of friends, and books chronicle the stories of fictional characters and real people alike~\cite{serban2016building}.
\section{Persona Characterization and Variation3}
\subsection{Background and Motivation}
The written text that we interact with on an everyday basis—news articles, emails, social media, books is the product of a profoundly social phenomenon with people at its core. With few exceptions, all of the text we see is written by people, and others constitute its audience. A vast amount of the content itself is centered on people: news (including classic NLP corpora such as the Wall Street Journal and the New York Times) details the roles of actors in current events, social media (including Twitter and Facebook) documents
the actions and attitudes of friends, and books chronicle the stories of fictional characters and real people alike~\cite{serban2016building}.
\subsection{Research Purpose}
The written text that we interact with on an everyday basis—news articles, emails, social media, books is the product of a profoundly social phenomenon with people at its core. With few exceptions, all of the text we see is written by people, and others constitute its audience. A vast amount of the content itself is centered on people: news (including classic NLP corpora such as the Wall Street Journal and the New York Times) details the roles of actors in current events, social media (including Twitter and Facebook) documents
the actions and attitudes of friends, and books chronicle the stories of fictional characters and real people alike~\cite{serban2016building}.
\subsection{Methodology}
The written text that we interact with on an everyday basis—news articles, emails, social media, books is the product of a profoundly social phenomenon with people at its core. With few exceptions, all of the text we see is written by people, and others constitute its audience. A vast amount of the content itself is centered on people: news (including classic NLP corpora such as the Wall Street Journal and the New York Times) details the roles of actors in current events, social media (including Twitter and Facebook) documents
the actions and attitudes of friends, and books chronicle the stories of fictional characters and real people alike~\cite{serban2016building}.
\section{Persona Characterization and Variation4}
\subsection{Background and Motivation}
The written text that we interact with on an everyday basis—news articles, emails, social media, books is the product of a profoundly social phenomenon with people at its core. With few exceptions, all of the text we see is written by people, and others constitute its audience. A vast amount of the content itself is centered on people: news (including classic NLP corpora such as the Wall Street Journal and the New York Times) details the roles of actors in current events, social media (including Twitter and Facebook) documents
the actions and attitudes of friends, and books chronicle the stories of fictional characters and real people alike~\cite{serban2016building}.
\subsection{Research Purpose}
The written text that we interact with on an everyday basis—news articles, emails, social media, books is the product of a profoundly social phenomenon with people at its core. With few exceptions, all of the text we see is written by people, and others constitute its audience. A vast amount of the content itself is centered on people: news (including classic NLP corpora such as the Wall Street Journal and the New York Times) details the roles of actors in current events, social media (including Twitter and Facebook) documents
the actions and attitudes of friends, and books chronicle the stories of fictional characters and real people alike~\cite{serban2016building}.
\subsection{Methodology}
The written text that we interact with on an everyday basis—news articles, emails, social media, books is the product of a profoundly social phenomenon with people at its core. With few exceptions, all of the text we see is written by people, and others constitute its audience. A vast amount of the content itself is centered on people: news (including classic NLP corpora such as the Wall Street Journal and the New York Times) details the roles of actors in current events, social media (including Twitter and Facebook) documents
the actions and attitudes of friends, and books chronicle the stories of fictional characters and real people alike~\cite{serban2016building}.
\section{Persona Characterization and Variation5}
\subsection{Background and Motivation}
The written text that we interact with on an everyday basis—news articles, emails, social media, books is the product of a profoundly social phenomenon with people at its core. With few exceptions, all of the text we see is written by people, and others constitute its audience. A vast amount of the content itself is centered on people: news (including classic NLP corpora such as the Wall Street Journal and the New York Times) details the roles of actors in current events, social media (including Twitter and Facebook) documents
the actions and attitudes of friends, and books chronicle the stories of fictional characters and real people alike~\cite{serban2016building}.
\subsection{Research Purpose}
The written text that we interact with on an everyday basis—news articles, emails, social media, books is the product of a profoundly social phenomenon with people at its core. With few exceptions, all of the text we see is written by people, and others constitute its audience. A vast amount of the content itself is centered on people: news (including classic NLP corpora such as the Wall Street Journal and the New York Times) details the roles of actors in current events, social media (including Twitter and Facebook) documents
the actions and attitudes of friends, and books chronicle the stories of fictional characters and real people alike~\cite{serban2016building}.
\subsection{Methodology}
The written text that we interact with on an everyday basis—news articles, emails, social media, books is the product of a profoundly social phenomenon with people at its core. With few exceptions, all of the text we see is written by people, and others constitute its audience. A vast amount of the content itself is centered on people: news (including classic NLP corpora such as the Wall Street Journal and the New York Times) details the roles of actors in current events, social media (including Twitter and Facebook) documents
the actions and attitudes of friends, and books chronicle the stories of fictional characters and real people alike~\cite{serban2016building}.

\bibliography{bibliography,custom}
\bibliographystyle{proposal_natbib}
\newpage
\begin{center}{\Large\bf Signature Page}\end{center}
\vspace{0.5cm}

% \vfill
\newcommand{\namesigdate}[2][5cm]{
\begin{minipage}{#1}
    #2 \vspace{1cm}\hrule\smallskip
\end{minipage}
}

\noindent \namesigdate[6cm]{Signature of domestic supervisor} \ \hfill \namesigdate[6cm]{Signature of hosting supervisor} 
\\
\vspace{.3cm}\smallskip \\ 
\noindent \namesigdate[6cm]{Date(yy/mm/dd)} \ \hfill \namesigdate[6cm]{Date(yy/mm/dd)}
\\
\end{document}
